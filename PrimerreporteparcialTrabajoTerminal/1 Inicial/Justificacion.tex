\section{Justificación}
A continuación, se presentan algunos de los motivos para elaborar un eje lineal compatible con el ambiente de ROS.\\
Robotic Operating System o ROS por sus siglas en inglés, es un compendio de librerías y herramientas de software libre que sirven para construir sistemas robóticos de todo tipo sin tener que “reinventar la rueda” pues ROS tiene paquetes para casi cualquier aplicación, desde software para simulaciones hasta herramientas de visión artificial y planeación de trayectorias. ROS empezó como un proyecto exclusivamente académico pero cada vez gana más terreno en aplicaciones industriales sin embargo todavía se requiere mejorar muchos aspectos y añadir elementos. \cite{Referencia6}.\\
\\Primero, es relevante mencionar la utilidad de los robots móviles o sobre vía como el que en este proyecto se busca desarrollar. La virtud principal de los robots móviles se encuentra en aquellas aplicaciones donde se requiere cubrir un área de trabajo amplia. Por ejemplo, en aplicaciones de soldadura, pintura o en el caso particular de los robots sobre ejes lineales, transporte de piezas donde se requiera mantener fija la posición relativa entre esta y el robot.
\\Se puede argumentar el uso de software de código abierto en este proyecto porque busca superar ciertas barreras de entrada a las tecnologías en vanguardia en la industria, especialmente en el apartado de software privativo que dificulta el aprendizaje y el desarrollo de nuevas tecnologías como menciona Richard Stallman \cite{Referencia1}: \say{...el software libre otorga a los usuarios la libertad de controlar sus propios ordenadores; con el software privativo, en cambio, el programa hace lo que el propietario o el programador quiere que haga, no lo que el usuario desea}. En este caso implica que los usuarios están restringidos a las herramientas de software que el fabricante proporciona, en el mejor de los casos estas herramientas se pueden descargar y utilizar sin costo alguno, pero están limitadas a el uso de robots del mismo fabricante. También existen herramientas de programación y simulación con compatibilidad para diferentes marcas de robots, sin embargo, las licencias de este tipo de software privativos llegan a ser una inversión inviable para un estudiante, micro o incluso pequeñas empresas, además de que se crea una dependencia hacia estos programas.\\ 
Estas limitaciones del software privativo llevan al empleo de alternativas de código abierto como lo es ROS. ROS nació como un proyecto académico para crear software para robots de propósito general, en un ambiente donde se pudiera colaborar y contribuir entre laboratorios o grupos de trabajo en diferentes partes del mundo, por ello se creó como un proyecto de código abierto aunque el mismo Stallman menciona que el código abierto y el código libre no son lo mismo \cite{Referencia2} \say{Ambas expresiones describen casi la misma categoría de software, pero representan puntos de vista basados en valores fundamentalmente diferentes. El código abierto es una metodología de programación, el software libre es un movimiento social}, para fines prácticos, ROS se puede considerar una alternativa de código abierto y código libre (contando incluso con la licencia BSD avalada por la Free Software Fundation creada por el mismo Stallman) a los simuladores y programas privativos de la industria robótica, además de que ROS cuenta con una amplia comunidad de desarrolladores e incluso el apoyo de empresas, por lo que es una alternativa igual de profesional. 
\\Este eje lineal para cobots contribuye un nodo nuevo a un framework de código abierto mundialmente conocido como lo es ROS, teniendo el potencial de beneficiar a cualquier estudiante, investigador o ingeniero que esté trabajando en la integración de un sistema con robots colaborativos ligeros, que se podrá ahorrar un tiempo significativo al proceso de desarrollo. En particular el proyecto puede beneficiar a pequeñas y medianas empresas de robótica en México a disminuir los tiempos y costos de desarrollo e investigación al tener disponible un prototipo de laboratorio funcional integrado con herramientas de software de fácil acceso y gratuitas.
\\El proyecto también busca beneficiar el ambiente académico, ya que experimentar con robots colaborativos es muy complicado debido a su alto coste, con este prototipo los estudiantes o investigadores tendrían acceso a más opciones para diseñar sistemas robóticos con una retroalimentación mediante simulaciones.
\\Por último, aunque la ingeniería mecatrónica está inmersa en cualquier proyecto de robótica, se debe resaltar su importancia en esta área. Especialmente porque hay que tomar en cuenta a todas las áreas de la mecatrónica para obtener un producto funcional y cumplir con los requerimientos de diseño como lo menciona Hagen Schempf \cite{Referencia3}, \say{[a] fin de desarrollar cualquier sistema robótico, ya sea tele operado [sic] o automatizado, se debe considerar que la robótica verdaderamente representa un área multidisciplinaria donde diferentes áreas de diseño (mecánico, eléctrico y de software) se deben integrar de manera exitosa en un complejo sistema.}
