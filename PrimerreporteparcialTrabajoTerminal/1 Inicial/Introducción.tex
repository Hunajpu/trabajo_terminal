\section{Introducción}
Un robot colaborativo o cobot es un robot manipulador típicamente de 6 grados de libertad, que tiene las características de seguridad adecuadas para laborar en conjunto con un trabajador humano, en contraste con los robots industriales tradicionales que se encuentran completamente aislados del contacto con humanos. Estrictamente para que un robot se considere colaborativo debe cumplir con la norma ISO/TS 15066\cite{cobots}.\\
Aunque el concepto de cobots es relativamente reciente, actualmente ya existen múltiples fabricantes que producen este tipo de robots y sus beneficios dentro de la industria son la flexibilidad, adaptabilidad y menor costo en comparación con los robots industriales clásicos, pudiendo realizar las mismas tareas que estos, aunque no se consideren colaborativas. Los robots colaborativos requieren el desarrollo de herramientas tecnologías complementarias similares a las que existen para los robots tradicionales como mesas posicionadoras, efectores finales especializados y ejes lineales horizontales y verticales.\\
Los robots colaborativos se comienzan a caracterizar por su fácil programación sin necesidad de requerir el desarrollo tradicional de códigos complejos, esto aunque una ventaja en ciertas aplicaciones sencillas, puede ser inconveniente para trabajos que requieran mayor complejidad, por ello en este proyecto se emplean herramientas de código abierto para programar los robots, y se ampliara el área de trabajo de los mismos con un séptimo grado de libertad prismático y así aprovechar todo el potencial de los robots colaborativos.
