\section{Planteamiento del problema}

La problemática que se pretenden trabajar es que los productos de desplazamiento lineal para cobots que se encuentran en el mercado tienen problemas de accesibilidad para pequeñas empresas y el sector académico. La mayoría están diseñados para los robots colaborativos de la marca Universal Robots (como se puede apreciar en la sección de Antecedentes), este proyecto por el contrario tiene como objetivo la compatibilidad con múltiples modelos de cobots.\\
La razón de que predominen los productos para los cobots de Universal Robots es que esta compañía ha creado una interfaz de programación de aplicaciones (API) popular, que permite crear hardware y software complementario de forma fácil, rápida, con opciones de depuración y simulaciones, además es accesible para cualquier usuario, por lo que las pequeñas y medias empresas de robótica se les facilita el desarrollo de productos para esta marca. Sin embargo, todos estos elementos complementarios creados por terceros solo son accesibles si se adquieren los productos, lo que quiere decir que no se podría hacer una evaluación previa (mediante simulaciones) de una celda robótica sin realizar este tipo de inversión. Este trabajo busca reducir esta problemática usando programas de código abierto y código libre, junto con un prototipo completamente funcional que justifique la funcionalidad del software que se pretende desarrollar.\\
Se perciben 2 desafíos principales para la elaboración de este trabajo el primero y más complicado de ellos es la sincronización del brazo robótico con el eje de movimiento que se plantea diseñar y construir, de tal forma que todo el sistema se perciba como un cobot de 7 grados de libertad, esta tarea es particularmente complicada porque se busca tener compatibilidad con cobots de diferentes marcas. El segundo reto de ingeniería es lograr un movimiento lineal preciso puesto que los robots montados sobre ejes lineales realizan movimientos repetibles pero de poca precisión como Eitel\cite{Referencia5} menciona.
