\section{Resumen}
En el presente trabajo se muestra las bases, desarrollo y elaboración de un eje lineal (séptimo grado de libertad) para un robot colaborativo ligero usando software libre. Por medio de este es posible trabajar de manera conjunta con el humano y permitir flexibilidad, adaptabilidad e incluso menos costo que los robots tradicionales. La investigación se conforma por diversos apartados que facilitan su comprensión. Para la introducción se explica a grandes rasgos la función y posibilidades del cobot para así aterrizar el planteamiento del problema, donde se explica la problemática detectada que da cabida este proyecto.\\
Posteriormente se ahonda en la justificación y pertinencia del trabajo a raíz del conflicto antes descrito. En seguida se encuentran los objetivos a perseguir a lo largo del escrito para así llegar a los antecedentes que marcan la pauta de las investigaciones y avances que ha habido al respecto del tema en los últimos años.\\
La metodología abarca los conceptos y diagramas teóricos mínimos para comprender el desarrollo del proyecto. Es así como la propuesta de solución estará diseñada en la metodología VDI 2206, basado en las necesidades, requerimientos, diagramas funcionales y su arquitectura física. En seguida es importante considerar también los recursos humanos, materiales y temporales que se emplearan para llevar a cabo todo lo anterior. \\
De esta manera se propone un proyecto de innovación, diversidad y con un enfoque mecatrónico para la industria de la robótica.