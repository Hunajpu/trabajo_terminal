\begin{thebibliography}{10} %10 significa el numero m�ximo de items
%Aqu� ponga la bibliograf�a y referencias usadas

%Art�culo:
\bibitem{cobots} Productive Robots, \say{COBOTS 101}, 2021. [En línea]. Disponible en: https://www.productiverobotics.com/cobots-101


\bibitem{Referencia5}
L. Eitel, \say{Designing (seventh-axis) linear motion tracks for robotic positioning}, 2016. [En línea]. Disponible en: https://bit.ly/3mE07Sh [Accedido: 10-oct-2021]

\bibitem{Referencia6}
Open Robotics, \say{ROS Concepts}, 2021. [En línea]. Disponible en: https://wiki.ros.org/ROS/Concepts [Accedido: 10-oct-2021]


\bibitem{Referencia1} R. Stallman, \say{Por qué las escuelas deben usar exclusivamente software libre}, 2021. [En línea]. Disponible en: https://www.gnu.org/education/edu-schools.es.html. [Accedido: 10-oct-2021]

%Libro
\bibitem{Referencia2}
R. Stallman, \say{Por qué el «código abierto» pierde de vista lo esencial del software libre}, 2021. [En línea]. Disponible en: https://www.gnu.org/philosophy/open-source-misses-the-point.es.html. [Accedido: 10-oct-2021]

\bibitem{Referencia3}
H. Schempf, \say{Mobile Robots and Walkign Machines}, en  \emph{Handbook of Industrial Robotics}, 2nd Ed., New York, USA, John Wiley \& Sons, Inc., 1999, pp. 145-165.

\bibitem{Referencia4}
NSK Ltd Precision Machinery \& Parts e-Proyect Team, \say{Linear Guides Tutorial}, 2021. [En línea]. Disponible en: https://www.bearing.co.il/Linear\_Guides\_Tutorial.pdf [Accedido: 10-oct-2021]

\bibitem{Referencia7}
VDI-Fachbereich Produktentwicklung, \say{Design methodology for mechatronic systems}, \textit{VDI-Gesellschaft Produktund Prozessgestaltung, Norme VDI}, vol. 2206, 2004.

\end{thebibliography}


